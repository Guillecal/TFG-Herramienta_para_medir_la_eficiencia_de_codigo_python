\capitulo{1}{Introducción}

La eficiencia en el mundo informático, es un termino fundamental y con mucho peso, que condiciona en la gran mayoría de ocasiones todo lo relacionado a este. Cuanta mayor eficiencia menos recursos se gastan para poder realizar un objetivo y estos recursos pueden ser utilizados para otras metas. 
La búsqueda de una buena eficiencia tiene una gran importancia dentro del ámbito laboral, donde cada recurso ahorrado puede ser fundamental, para el desarrollo de otras tareas. (En sistemas de tiempo real es algo fundamental)\\

Aunque la la eficiencia es algo que se puede aplicar a casi cualquier campo conocido, en este caso nos vamos a centrar meramente en la eficiencia aplicada a código Python, ya que es el tema principal de este trabajo. 
Para poder medir la eficiencia de un código hecho en lenguaje Python, primero se tendrá que empezar por determinar y desarrollar que es lo que hace que un código sea mas eficiente que otro y el como conseguir esa información.\\

A la primera pregunta la respuesta seria, "depende". Ya que según el entorno y el objetivo para el que ha sido creado un programa, lo que determina si un programa esta hecho de forma eficiente puede variar entre diferentes métricas. Para establecer un forma de medir la eficiencia en esta herramienta se ha decidido utilizar las operaciones como forma primaria de obtención de información.\\ 

Hay que dejar claro que hay mas formas de medir la eficiencia de un programa como por ejemplo:
\begin{itemize}
	\item Medir el tiempo de ejecución: Esta es la forma de medirla eficiencia mas utilizada, esto se debe a la facilidad de llevarla a cabo, pero como gran inconveniente se podría decir que es una manera que devuelve resultados muy poco afinados. Esto es por la gran influencia que  tiene algunos factores  externos en los resultados que puede devolver. El mayor de estos factores seria el entrono de pruebas donde se ejecuta el código del programa, ya que si hacemos una ejecución en dos ordenadores totalmente diferentes, es muy probable que algoritmos con una cierta complejidad  tarde mas en ser ejecutados  en un equipo que en otro. Con este  método  para poder sacar resultados con una gran fiabilidad habría que tener un gran control sobre el entorno, lo cual requiere de una gran preparación previa.
	\item Complejidad O: Esta forma realmente nos sirve  para conocer la complejidad del código, pero también es a veces es tomado como referencia para hacerse una estimación de la eficiencia. Al ser tan solo una estimación no se puede realizar un análisis en profundidad sobre este aspecto.

\end{itemize}



 
Pero se quiere implementar la medición con operaciones por ser una  manera con soluciones invariables sea cual sea el entono en el que se haga y devuelve datos con una cierta profundidad como para poder hacer todos los análisis que se requieran.\\

En el caso de este trabajo se ha decidido que para conseguir la información de cuantas operaciones hay en la ejecución de un programa se utilizara un interprete, que como bien su nombre indica, sera el responsable de interpretar el código que forma el programa y tras hacerle unas modificaciones nos guardara  la información que deseada. 



