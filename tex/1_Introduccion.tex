\capitulo{1}{Introducción}

La eficiencia en el mundo informatico, es un termino fundamental y con mucho peso, que condiciona en la gran mayoria de ocasiones todo lo relacionado a este. Cuanta mayor eficiencia menos recursos se gastan para poder realizar un objetivo y estos recursos pueden ser utilizados para otras metas. 
La busqueda buena efciciencia tiene una gran importancia dentro del ambito laboral, donde cada recurso ahorrado puede ser fundamental, para el desarrollo de otras tareas.
En este caso nos vamos a centrar meramente en la eficiencia aplicada a codigo Python, ya que es el tema principal de este trabajo y la eficiencia, es algo que se puede aplicar a casi cualquier campo conocido. 
Para poder medir la eficiencia de un codigo hecho en lenguaje Python, primero se tendrar que empezar por determinar y desarrollar que es lo que hace que un codigo sea mas eficiente que otro y el como conseguir esa informacion.
A la primera pregunta la respuesta seria, "depende". Segun el entorno y el objetivo para el que ha sido creado un programa, lo que determine si ese programa esta hecho de forma eficiente puede variar entre diferentes metricas. Para estableder un forma de medir la eficiencia en esta herramineta se ha decidido utilizar las operaciones como forma primaria de obtencion de informacion. Mas adelante estas operacion serviran para conseguir una 
Y para conseguir la informacion de cuantas operaciones hay en la ejecución de un programa se utilizara un interprete, que como bien su nombre indica sera el responsable de interpretar el codigo que forma el programa.


