\capitulo{1}{Introducción}

La eficiencia en el mundo informatico, es un termino fundamental y con mucho peso, que condiciona en la gran mayoria de ocasiones todo lo relacionado a este. Cuanta mayor eficiencia menos recursos se gastan para poder realizar un objetivo y estos recursos pueden ser utilizados para otras metas. 
La busqueda buena efciciencia tiene una gran importancia dentro del ambito laboral, donde cada recurso ahorrado puede ser fundamental, para el desarrollo de otras tareas. (En sistemas  de tiempo real ni te  cuanto)\\

En este caso nos vamos a centrar meramente en la eficiencia aplicada a codigo Python, ya que es el tema principal de este trabajo y la eficiencia, es algo que se puede aplicar a casi cualquier campo conocido. 
Para poder medir la eficiencia de un codigo hecho en lenguaje Python, primero se tendrar que empezar por determinar y desarrollar que es lo que hace que un codigo sea mas eficiente que otro y el como conseguir esa informacion.\\

A la primera pregunta la respuesta seria, "depende". Segun el entorno y el objetivo para el que ha sido creado un programa, lo que determine si ese programa esta hecho de forma eficiente puede variar entre diferentes metricas. Para estableder un forma de medir la eficiencia en esta herramineta se ha decidido utilizar las operaciones como forma primaria de obtencion de informacion.\\ 

Hay que dejar claro que hay mas formas de medir la eficiencia de un programa como por ejemplo:
\begin{itemize}
	\item Medir el tiempo de ejecucion: Esta es la formade medirla eficiencia mas utilizada, esto se debe a la facilidad de llevarla a cabo. Pero como gran incomveniente se podria decir que es una manera que devuelve resultados muy poco afinados. Esto es por la gran influencia que  tiene algunos factores  externos en los resultados que puede salir. El mayo  de estos factores seria el entrono de pruebas donde se ejecuta el código del programa, ya que si hacemos una ejecución en dorordenadores totalmente diferentes es muy probable  que algoritmos con una cierta complejidad  tarde mas en ser ejecutados  en un equipo que en otro. Con este  metodo  para poder sacar resultados con una gran fiabilidad habria que tener un gran control sobre el entorno, lo cual requiere de un ad 
	\item Complejidad O: Esta forma realmente nos sirve  para conocer la commplegidad del codigo, pero tambien es a veces es tomado como referencia para hacerse una estimacion de la eficiencia. Al ser tan solo una estimacion no se puede realizar un analisis en profundidad sobre este aspecto.

\end{itemize}



 
Pero se quiere implementar la medicion con operaciones por ser una  manera con soluciones invariables sea cual sea el entono en el que se haga y dado datos con una cierta profundidad para poder hacer todos los analisis que se requieran.\\

En el caso de este trabajo se ha decidido que para conseguir la información de cuantas operaciones hay en la ejecución de un programa se utilizara un interprete, que como bien su nombre indica, sera el responsable de interpretar el código que forma el programa y tras hacerle unas modificaciones nos guardara  la información que deseada. 



