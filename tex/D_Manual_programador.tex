\apendice{Documentación técnica de programación}



\section{Introducción}
En esta sección se explicaran los conceptos necesarios para poder ponerse a trabajar con este proyecto:\\

Lo primero antes de nada seria descargarse la herramienta desde aquí:\\

Se tiene que descargar desde: (url)\\



\section{Estructura de directorios}
Los archivos necesarios están repartidos entre dos carpetas:\\

Vent.py\\
Resto de Interprete\\

\section{Manual del programador}
Se empezara preparando el entorno de trabajo para trabajar con el proyecto:\\

\subsection{Python}
Para el desarrollo de la herramienta se utilizó la versión de Python 2.7.X, es recomendable descargar esta versión para evitar algún tipo de incoherencia.\\

Link de descarga:\\

Versión 
Seguido a esto es necesario instalar las bibliotecas utilizadas dentro del programa. Por ello es necesario instalar primero el modulo?¿?¿ pip, el cual es esencial para hacer las instalaciones.
Para este proyecto se utilizó la versión xxxxx, pero en este caso la versión no debería afectar, es recomendable descargar la versión más reciente:\\

Link Descarga\\

Cuando ya se tiene pip instalado ya solo sería meter os siguientes comandos por el símbolo de sistema (Línea de comandos):\\

 (Con sus diversas librerías)
\subsection{IDE}
Como entorno de desarrollo se podría se puede utilizar el IDE de Python, pero para el desarrollo de la herramienta se ha utilizado Spyder. \\
Link descaga:\\
Una vez descargado e instalado este IDE ya viene listo para empezar poder trabajar, pero en caso de que se requiera, se puden cambiar algunas configuraciones según el gusto de cada uno desde la pestaña ….\\
Imagen de la interfaz de Spyder\\
Esta es una recomendación, si se está acostumbrado utilizar otro tipo de IDE que sirva para el lenguaje Python no habría ningún problema.\\

\subsection{Github Desktop}

Esta es la herramienta utilizada para gestionar mejor el repositorio donde se encuantra el trabajo
Se puede descargas desde aquí:\\

Una vez descargado e instalado, la primera vez que lo ejecutamos, podemos clonar desde aquí el proyecto desde el repositorio

\section{Compilación, instalación y ejecución del proyecto}
No se puede compilar al ser Python el lenguaje del código, ya que este es un lenguaje interpretado. Tampoco es necesaria una instalación, por cual lo único que queda por hacer es ejecutar el código.
Hay dos maneras de hacer esto:\\

A través de la línea de comandos\\
A través de un archivo ejecutable\\

\subsection{Linea de comandos}
Esta forma es muy sencilla. Simplemente hay abrir el Simbolo de sistema(Línea de comandos) y navegar hasta el directorio donde se encuentra el archivo Vent.py, esto se puede hacer fácilmente con el comando CD para cambiar de directorio y dir para mostrar los que contiene el directorio actual.\\
Una vez encontrado el directorio simplemente hay que ejecutar el comando: Python Vent.py\\

\subsection{Archivo ejecutable}
Simplemente seria clicar en el archivo .exe que se encuentra dentro de la carpeta dist.
Pero para hacer crear este documento es necesario previamente haber configurado un fichero llamado ....\\

Para sacar el fichero excel bastantes problemas con el fihcero .dll, hecho a partir de py2exe configurando setup.py
\section{Pruebas del sistema}
Pruebas del sistema:\\
Videos? Recalcar archivos con ejemplos?