\capitulo{7}{Conclusiones y Líneas de trabajo futuras}

Para terminar se expondrán las conclusiones obtenidas tal el desarrollo de todo este proyecto y las posibilidades que deja abierta para desarrollar mas cosas en el futuro:
\section{Conclusiones}
\begin{itemize}
	\item Se consiguieron realizar los objetivos principales propuestos para este proyecto. Entre estos están la implementación de una forma de medir la eficiencia de una manera flexible, con el propósito de que sea útil en un mayor numero de casos de usos.
	\item La labor de comprensión realizada para entender un código hecho por otras personas, en este caso concreto para entender el funcionamiento del ByteRun, fue más duradero de lo esperado ya que a pesar de no ser un interprete muy complejo tenía una serie de conceptos sobre las pilas y el Bytecode, al final fue una buena manera de aprender algo más sobre los diferentes tipos de interpretes que había y en concreto saber más sobre los interpretes de pila.
	\item Otro gran reto de esta aplicación fue la interfaz gráfica, ya que casi no había tenido experiencia creando una. Tanto elegir las herramientas necesarias para hacerla como el posterior aprendizaje sobre el funcionamiento de Tkinter, es otra de las cosas que me llevo aprendidas tras este proyecto
Aunque no todas las ideas que se plantearon para este proyecto salieron finalmente a la luz, ha sido muy satisfactorio y enriquecedor, ver como ideas que vas planteando se van plasmando poco a poco en algo que vas invirtiendo horas para finalmente formar aquello que estabas intentando lograr. Personalmente también me ha servido para aprender sobre la gestión que hay que llevar en un proyecto de estas características y como reaccionar ante los diferentes obstáculos que uno se va encontrando en el desarrollo del mismo.
\end{itemize}


\section{Líneas de trabajo futuras}
Este proyecto tiene un fuerte potencial para seguir siendo desarrollado, ya que a partir de este punto puede evolucionar hacia diferentes frentes:\\

En la herramienta se han implementado aquellos análisis que se han creído mas interesantes en una primera instancia que pueden ser usado en una ámbito mas general. Pero se pueden implementar más análisis dependiendo del objetivo que se tenga detrás, por poner un ejemplo, un análisis que estuvo a punto de ser implementado también en este proyecto fue, el análisis de operaciones individuales. Este análisis consistiría en mostrar como aumenta la cantidad de un tipo de operador en concreto cada vez que el interprete encuentra un operador. Este análisis sería una buena forma de ver en que parte de la ejecución un código se aglomeran mas tipos de operaciones. Y este es solo un ejemplo de los muchos tipos de análisis más que se pueden implementar.\\

Y en lo que más potencial creemos que puede derivar en un futuro esta herramienta sería el poder medir la eficiencia de códigos de casi cualquier tipo, esto sería relativamente sencillo tras haber desarrollado esta herramienta, debido a la forma interna en la que esta construida . La interfaz está separada del resto del programa y gracias a esto se podrían añadir interpretes que pudieran analizar tipos de lenguaje de programación, y al resto de la herramienta habría que hacerle unos cambios mínimos para que permitiese la entrada de ficheros con otro tipo de formato que no sea solamente .py. Esto haría que la herramienta fuese útil para casi cualquier tipo de desarrollo.\\

Como últimos planteamiento también se podría mirar la migración de version, de la version 2.7 de Python a alguna de las versiones 3.
Algunos elementos de la interfaz podrían se mejorables como el evitar la superposición de los nombres de los tipos de operaciones detectados en el análisis individual cuando se trata de un fichero con muchos tipos.

