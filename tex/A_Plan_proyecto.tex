\apendice{Plan de Proyecto Software}

\section{Introducción}
En esta sección se plantea la forma de estimar los costes del proyecto de la manera mas precisa. Los costes se dividen en tres: costes de tiempo, de trabajo y monetarios.

Se determinan los recursos necesarios para llevar este proyecto a un entorno laboral.


\begin{itemize}
	\item Planificación temporal: Esta planificación sirve para ajustar los tiempos que tiene que tardar cada parte del proyecto. El tiempo que tarde cada parte puede variar según el nivel de complejidad y la cantidad de tareas que tenga.
	\item Latex: Estudio de viabilidad: Esta planificación es el que determina si el proyecto se podría trasladar al marco laboral. Muestra los costes monetarios y las licencias a las que este sujeto el proyecto.
\end{itemize}





\section{Planificación temporal}
Para el desarrollo del proyecto se utiliza la metodología Kanban para organizar el flujo de trabajo, pero este solo es una manera mas visual de poder mantener un buen rutmo de trabajo, para saber que hacer en cada etapa del proyecto se identifican unas .

Las parte en las que se ha dividido el desarrollo del proyecto son estas 4:

Investigación:
Es la etapa mas temprana del proyecto, aqui es cuando hay que recoger información


Desarrollo:


Pruebas: 


Documentación:

\section{Estudio de viabilidad}

\subsection{Viabilidad económica}
Aquí se plantean los constes y beneficios en el caso de que el proyecto se realizase en un entorno laboral.

Los costes se dividen en 3 tipos:


Coste de personal

Son los costes que supone tener a  un desarrollador contratado en el 2019, teniendo en cuenta tanto su salario neto, la cortización a la seguridad social y la retencion del IRPF.

\begin{table}[htbp]
\begin{center}
\begin{tabular}{|l|l|}
\hline
Concepto & Coste ()\\
\hline \hline
Salario neto & int \ \hline
Seguridad social & flo  \\ \hline
IRPF & int \\ \hline
Salario bruto & flo \\ \hline
Total al mes & int \\ \hline
\end{tabular}

Costes de infraestructura hardware
En el caso de este proyecto el coste en hardware tan solo sería para costear un ordenador donde poder  desarrollar la herramienta.

\begin{table}[htbp]
\begin{center}
\begin{tabular}{|l|l|l|}
\hline
Concepto & Coste () & INT\\
\hline \hline
ADD & int & 100\\ \hline
\end{tabular}

Costes de infraestructura software
Estos costes son los que supone las licencias del software que se necesita para hacer el desarrollo del proyecto. En caso de este proyecto tan solo habría que costear una licencia del sistema operativo.

\begin{table}[htbp]
\begin{center}
\begin{tabular}{|l|l|l|}
\hline
Concepto & Coste () & INT\\
\hline \hline
Windows 10 Pro & 259 & 100\\ \hline
\end{tabular}

Beneficios

\subsection{Viabilidad legal}
Se han investigado las licencias de las herramientas utilizadas en el desarrollo para saber la  viabilidad del proyecto. He aquí las licencias:

\begin{itemize}
	\item MatPlotLib: BSD.
	\item Latex: BSD.
	\item Spyder: BSD.
	\item Mostrar esa información de una manera agradable y entendible.
\end{itemize}

Como se ve en el listado anterior ninguna de las licencias es un impedimento para la distribución del proyecto.


