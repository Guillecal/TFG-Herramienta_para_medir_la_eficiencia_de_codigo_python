\capitulo{2}{Objetivos del proyecto}

\section{Objetivos del Proyecto}

\subsection{Objetivo general}


Desarrollar un software capaz de dar la información necesaria para realizar un análisis de eficiencia sobre un código de extension .py.
\begin{itemize}
	\item Poder hacer análisis de un código python a través del calculo de operaciones.
	\item Mostrar esa información de una manera agradable y entendible.
\end{itemize}


\subsection{Objetivos funcionales}
	\begin{itemize}
	\item El usuario podrá elegir el tipo de análisis que quiera realizar: de un fichero individual o comparando varios fichero.
	\item El usuario podrá modificar la información resultante, eligiendo las operaciones que saldrán.
	\item El usuario puede cambiar la métrica de los análisis, decidiendo como pondera cada tipo de operación
	\item El usuario podrá realizar varios análisis sin necesidad de salir de la herramienta
	\item El usuario podrá hacer una parametrización a través de los  nombres de los ficheros
\end{itemize}


\subsection{Objetvios técnicos}
	\begin{itemize}
	\item Obtener el numero de operaciones a través del análisis por parte del interprete
	\item Diseñar una manera de guardas los niveles de eficiencia de cada tipo de operación.
	\item Obtener los datos finales con los que realizar la  gráficas a través de la ponderación
	\item Utilizar Tkinter para realizar la interfaz de la herramienta
	\item Utilizar Matplotlib  para generar las gráficas tras los respectivos análisis
	\item funcionamiento de ByteCode en Python
\end{itemize}
