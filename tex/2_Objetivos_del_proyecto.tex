\capitulo{2}{Objetivos del proyecto}

\section{Objetivos del Proyecto}

\subsection{Objetivo general}


Desarrollar un software capaz de dar la informacion necesaria para realizar un analisis de eficiencia sobre un codigo de extension .py
\begin{itemize}
	\item Poder hacer analisis de un codigo python independientemente del contenido de este.
	\item Mostrar esa informacion de una manera agradable y entendible.
\end{itemize}


\subsection{Objetivos funcionales}
	\begin{itemize}
	\item El usuario podra elegir el tipo de analisis que quiera realizar: de un fichero individual o comparando varios fichero.
	\item El usuario podra modificar la informacion resultante, eligiendo las operaciones que saldran.
	\item El usuario puede cambiar la metrica de los analisis, decidiendo como pondera cada tipo de operacion
	\item El usuario puede cambiar los parametros de entrada de algunos programas para ver su la eficiencia en diferentes x
	\item El usuario podra realizar varios analisis sin necesidad de salir de la herramienta
	\item(Exportar la informacion?)
\end{itemize}


\subsection{Objetvios tecnicos}
	\begin{itemize}
	\item Obtener el numero de operaciones a traves del analisis por parte del interprete
	\item Diseñar una manera de guardas los niveles de eficiencia de cada tipo de operación.
	\item Obtener los datos finales con los que realizar la  graficas a traves de la ponderacion
	\item Utilizar Tkinter para realizar la interfaz de la herramienta
	\item Utilizar xxx  para generar las graficas tras los respectivos analisis
	\item alguan erramienta mas?
\end{itemize}
