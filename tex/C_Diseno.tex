\apendice{Especificación de diseño}

\section{Introducción}
En esta sección se explican todos los aspectos relevantes sobre el diseño de la herramienta como el diseño de la arquitectura o los tipos de datos que se utilizan.

\section{Diseño de datos}
Los datos que se utilizan en la herramienta son los siguientes:
\begin{itemize}
	\item Archivos.py: Se utilizan los códigos hechos en lenguaje python de estos archivos para hacer su análisis.
	\item Archivos.csv: Se forma una matriz dentro de estos archivos para hacer la ponderación de operaciones. 
	\item ByteCode: Es el código intermedio que surge tras compilar el codigo de alto  nivel de los archivos.py.
	
\end{itemize}


\section{Diseño procedimental}
A continuación se verán las interacciones entre los elementos de la herramienta:

\begin{figure}[H]
\centering
\includegraphics[width=9cm, height=12cm]{Secuencia}
\caption{Diagrama secuencial de la ejecución de la herramienta}
\end{figure}



Como se puede ver en el diagrama, el usuario tiene una interacción constante con la herramienta en toda su ejecución.
\section{Diseño arquitectónico}
El diseño arquitectónico de la herramienta se basa en un sistema de ventanas que se llaman las unas a la otras y mientras tanto hay un controlador que gestiona todas estas llamadas.
También hay ciertas ventanas en concreto que son las encargadas de hacer llamadas al Intérprete cuando es  necesario las cuales son:

\begin{itemize}
	\item VentanaAnalisis.
	\item VentanaMultiple. 	
\end{itemize}

