\capitulo{6}{Trabajos relacionados}

El tema tratado en este trabajo ha sido tratado desde ya  hace bastante tiempo, ya que la eficiencia un campo que lleva teniendo importancia ya años atrás, a la par de que es posible abordarla desde un gran numero de puntos de vista diferentes.\\

Aquí se hará mención a un par de trabajos o proyectos que tengan que ver con mas con la eficiencia en el campo de la informática y se consideren interesantes para combinar y ampliar los conocimientos expuestos en este trabajo.\\

Vamos a mencionar un articulo que explica diferentes formas de medir la eficiencia de codigo para codigo java de android, este es un trabajo que semuy  interesante que puede servir para enteder otro tipo  de medicion y enfoncado para un entorno mas expecifico como son las aplicaciones moviles.\\

Enlace: \url{https://ieeexplore.ieee.org/abstract/document/7062696/keywords#keywords} 

(Hago referencia en bibliografia?¿)\\

Tambien se va a destacar el siguiente proyecto:\\

Enlace: \url{https://github.com/react-rpm/react-rpm}

En este proyecto se ha creado una herramienta  para Chrome capaz de medir la eficiencia de aplicaciones Rect. Destacamos este proyecto por la gran cantidad de opciones que posee y algunas funcionalidades, commo por ejemplo el hecho de que puede enseñar información en tiempo real según la va analizando.\\


A parte de lo mencionado anteriormente también se han encontrado algunos proyectos mas que tratan la medicion de eficiencia en diferentes lenguajes y pueden ser interesantes:\\
\begin{itemize}
	\item \url{https://github.com/DaKnOb/Zipper}.
	\item \url{https://github.com/MobileComputing2016/EfficiencyMeasurement}
\end{itemize}


