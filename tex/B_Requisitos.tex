\apendice{Especificación de Requisitos}

\section{Introducción}
En esta sección se van a explicar los diferentes objetivos de la herramienta junto sus respectivos requisitos.

\section{Objetivos generales}
Los objetivos generales de la herramienta son:

\begin{itemize}
	\item Recoger la elección del usuario sobre el tipo de análisis y los  ficheros que desea analizar.
	\item Identificar los diferente tipos de operaciones al ejecutar los ficheros a través del intérprete. 
	\item Enseñar los resultados  de los análisis a través de gráficas.
\end{itemize}

\section{Catalogo de requisitos}
Ahora se mostraran los requisitos, tanto funcionales como no funcionales, de los objetivos generales de la herramienta:

\subsection{Requisitos funcionales}
\begin{itemize}
	\item Encender la herramienta.
	\item Dejar elegir ficheros  a traves del explorador de archivos. 
	\item .
\end{itemize}
\subsection{Requisitos no funcionales}
\begin{itemize}
	\item Recoger la elección del usuario sobre el tipo de análisis y los  ficheros que desea analizar.
	\item Identificar los diferente tipos de operaciones al ejecutar los ficheros a través del intérprete. 
	\item Enseñar los resultados  de los análisis a través de gráficas.
\end{itemize}
\section{Especificación de requisitos}


