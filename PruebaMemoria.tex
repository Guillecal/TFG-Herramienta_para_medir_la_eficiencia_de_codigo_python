\documentclass[12pt,a4paper]{article}
\usepackage[latin1]{inputenc}
\usepackage[spanish]{babel}
\usepackage{amsmath}
\usepackage{amsfonts}
\usepackage{amssymb}
\usepackage{graphicx}
\usepackage[left=2cm,right=2cm,top=2cm,bottom=2cm]{geometry}
\author{Guillermo Calvo �lvarez}
\title{Memoria TFG}
\date{\today}

\begin{document}
\maketitle
%Primera prueba de \LaTeX{}
\tableofcontents
\newpage
\section{Decisiones Previas}

Dejare indicado en esta parte del documento aquellos detalles que he decidido tomar antes de la realizacion de TFG.

\subsection{Herramientas}
Para crear la documentacion y memoria del TFG, utilizare la distribucion de latex MikTeX 2.9.6 con el editor de texto TexMaker.

\subsection{Metodolog�a de trabajo}
Para seguir un flujo de trabajo constante y gestionar el trabajo he decidido utilizar el m�todo kanban, ya que me parece un m�todo bastante sencillo de llevar a cabo y es que creo que mejor se ajusta a el tipo de desarrollo que utilizare en este trabajo.

\subsection{Informaci�n previa}
Utilice la siguiente informacion sobre el bytecode:


\end{document}