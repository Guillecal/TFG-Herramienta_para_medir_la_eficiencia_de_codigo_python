\documentclass[12pt,a4paper]{article}
\usepackage[latin1]{inputenc}
\usepackage[spanish]{babel}
\usepackage{amsmath}
\usepackage{amsfonts}
\usepackage{amssymb}
\usepackage{graphicx}
\usepackage[left=2cm,right=2cm,top=2cm,bottom=2cm]{geometry}
\author{Guillermo Calvo �lvarez}
\title{Memoria TFG}
\date{\today}

\begin{document}
\maketitle
%Primera prueba de \LaTeX{}
\tableofcontents
\newpage
\section{Introduccion}

El objetivo de este TFG es el de desarrollar una herramienta capaz de analizar la eficiencia de un c�digo hecho en lenguaje Python e indicar posibles mejoras en este.

\section{Decisiones Previas}

Dejare indicado en esta parte del documento aquellos detalles que he decidido tomar antes de la realizacion de TFG.

\subsection{Herramientas}
Para crear la documentaci�n y memoria del TFG, utilizare la distribuci�n de latex MikTeX 2.9.6 con el editor de texto TexMaker. 
Para desarrollar el c�digo de la herramienta utilizare el entorno de desarrollo de anaconda (jupyter/spider)

\subsection{Metodolog�a de trabajo}
Para seguir un flujo de trabajo constante y gestionar el trabajo, he decidido utilizar el m�todo kanban, ya que me parece un m�todo bastante sencillo de llevar a cabo y es el que creo que mejor se ajusta a el tipo de desarrollo que utilizare en este trabajo.

En este caso he destaco dos figuras que influyen en el flujo de trabajo y son:
El desarrollador: en este caso soy yo el que ocupa este papel, se trata de la persona que va realizando las tareas que estan en el estado "to do" y se encarga de pasarla al estado "in progess" y al final en "done".\\
El Planificador: Esta papel lo lleva a cabo mi tutor?�?�?�?�?, el cual se encarga de ir seleccionando cuales son las tareas que deben ponerse con estado "to do", y asesoramiento de cual seria la mejor forma de afrontar estas tareas.

------	Comentar las reuniones semanales -------------------

\subsection{Informaci�n previa}

Utilice la siguiente informaci�n sobre el bytecode:\\
https://pymotw.com/2/dis/\\
http://www.aosabook.org/en/500L/a-python-interpreter-written-in-python.html\\
http://akaptur.com/blog/2013/08/14/python-bytecode-fun-with-dis/\\

\section{Tipos de Operaciones}
\begin{enumerate}
\item Suma
\item Resta
\item Multiplicaci�n
\item Divisi�n (/,//)
\item Potencia
\item Modulo
\end{enumerate}
\end{document}



